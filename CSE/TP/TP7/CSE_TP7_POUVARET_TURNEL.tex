\documentclass{article}

\usepackage[utf8]{inputenc}
\usepackage[T1]{fontenc}
\usepackage[francais]{babel}
\usepackage{url}
\usepackage{color}
\usepackage{verbatim}
\usepackage{amsmath,amssymb,amsfonts}
\usepackage{graphicx}
\usepackage[french]{algorithm2e}
\usepackage{geometry}
\usepackage{enumitem}
\usepackage{listings}
\usepackage{listingsutf8}
\usepackage{caption}
\captionsetup[figure]{slc=on}
\frenchbsetup{StandardLists=true}
\lstset{language=Java}
\lstset{
	breaklines=true, 
	showspaces=false, 
	keepspaces=true, 
	numbers=left, 
	frame=single, 
	keywordstyle=\color{blue},
	basicstyle=\ttfamily\small,
	commentstyle=\color{green}
}
\geometry{hmargin=2.5cm, vmargin=2.5cm}

\title{Conception des Systèmes d'Exploitation\\Rapport sur les performances de l'allocateur mémoire}
\author{Line \bsc{POUVARET}, Mickaël \bsc{TURNEL}}
\date{2015-2016}

\begin{document}
\maketitle
\section{Présentation du plan d'expérience}
Dans un premier temps, nous avons choisi d'effectuer des mesures sur le total de mémoire demandé par l'utilisateur (donc jusqu'à la dernière zone libre de la mémoire) et le total de mémoire utilisé (cumul des tailles des blocs occupés de mémoire).
\section{Résultats obtenus}
 
\section{Analyse des résultats}

\end{document}