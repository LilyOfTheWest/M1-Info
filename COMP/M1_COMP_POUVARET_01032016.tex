\documentclass{article}

\usepackage[utf8]{inputenc}
\usepackage{geometry}
\usepackage{indentfirst}

\geometry{hmargin=2.5cm,vmargin=2.5cm}

\title{Complexité}
\author{Line POUVARET\\Groupe 2}
\date{01/03/2016}

\begin{document}
\maketitle

\section*{\underline{Trouver une machine de Turing pour $\lambda$xy[x - y]}}

\subsection*{\textbf{Entrée :}}
2 entiers : x et y sur la bande représentés en unaire et séparés par un blanc : \\

$\rightarrow$ $|^{x+1}B|^{y+1}$

\subsection*{Sortie :}
$\left\{
	\begin{array}{ll}
		|^{x-y} ~ si ~ x \geq y \\
		\uparrow sinon
	\end{array}
\right.$

\subsection*{Principe :}

On parcourt chaque symbole des deux entiers alternativement et on remplace les $|$ trouvés par *.

Si l'entier x, situé à gauche sur la bande, n'a plus de $|$ mais que y, situé à droite, en a toujours, alors on boucle.

Sinon la machine affiche le résultat sur la bande.

\subsection*{Spécifications :}
\begin{itemize}
\item Tant qu'on trouve des $|$ pour l'entier de droite:
	\begin{itemize}
		\item On se déplace jusqu'au dernier $|$ de l'entier de droite
		\item On remplace $|$ par *
		\item On se déplace à gauche
		\item Si on lit un blanc alors il n'y a plus de $|$ pour l'entier de droite :
			\begin{itemize}
				\item S'il n'y a pas de $|$ dans l'entier de gauche alors on boucle (x = 0, x $<$ y)
				\item Sinon, on se déplace jusqu'au premier $|$ de l'entier de gauche, on le remplace par * et on s'arrête
			\end{itemize}
		\item Sinon, on se déplace jusqu'au premier $|$ de l'entier de gauche
		\item Si on trouve bien un $|$ dans l'entier de gauche, on le remplace par * et on recommence depuis le début
		\item Sinon on boucle car x $<$ y
	\end{itemize}
\end{itemize}

\subsection*{Machine de Turing (quadruplets) :}

\# On se déplace jusqu'au dernier $|$ de l'entier y

\begin{tabular}{llll}
$q_0$ & $|$ & R & $q_0$ \\
$q_0$ & B & R & $q_1$ \\
$q_1$ & $|$ & R & $q_1$ \\
\end{tabular}\\

\# On se déplace à gauche si on lit un blanc ou *

\begin{tabular}{llll}
$q_1$ & * & L & $q_2$ \\
$q_1$ & B & L & $q_2$
\end{tabular}\\

\# On remplace $|$ par *

\begin{tabular}{llll}
$q_2$ & $|$ & * & $q_2$
\end{tabular}\\

\# On se déplace à gauche (toujours dans l'entier y)

\begin{tabular}{llll}
$q_2$ & * & L & $q_3$
\end{tabular}\\

\# Si on lit un blanc alors il n'y a plus de $|$ sur l'entier de droite alors on passe dans un état spécial $q_{y0}$ 

\begin{tabular}{llll}
$q_3$ & B & L & $q_{y0}$
\end{tabular}\\

\# Sinon, on se déplace à gauche jusqu'à trouver un blanc (délimiteur)

\begin{tabular}{llll}
$q_3$ & $|$ & L & $q_4$ \\
$q_4$ & $|$ & L & $q_4$ \\
$q_4$ & B & L & $q_5$
\end{tabular}\\

\# Si on est dans l'état spécial et qu'on lit un * alors il n'y a pas de $|$ dans l'entier de gauche donc on boucle (x = 0, x $<$ y)

\begin{tabular}{llll}
$q_{y0}$ & * & * & $q_{y0}$
\end{tabular}\\

\# Sinon, si on lit un $|$ on le remplace par * et on s'arrête

\begin{tabular}{llll}
$q_{y0}$ & $|$ & * & $q_{fin}$
\end{tabular}\\

\# Sinon on se déplace tout à gauche jusqu'au premier blanc ou premier * rencontré et on se positionne à sa droite

\begin{tabular}{llll}
$q_5$ & $|$ & L & $q_5$ \\
$q_5$ & B & R & $q_6$ \\
$q_5$ & * & R & $q_6$
\end{tabular}\\

\# Si on trouve un $|$ on le remplace par * et on recommence depuis le début

\begin{tabular}{llll}
$q_6$ & $|$ & * & $q_6$ \\
$q_6$ & * & R & $q_0$
\end{tabular}\\

\# Sinon, on ne trouve plus de $|$ dans l'entier de gauche donc on boucle (x $<$ y)

\begin{tabular}{llll}
$q_6$ & B & B & $q_6$
\end{tabular}\\

\end{document}