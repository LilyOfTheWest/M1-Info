\documentclass{report}

\usepackage[utf8]{inputenc}
\usepackage{geometry}
\usepackage{indentfirst}

\geometry{hmargin=2.5cm,vmargin=2.5cm}

\title{Complexité}
\author{Mickaël Turnel\\Groupe 2}
\date{01/03/2016}

\begin{document}
\maketitle

\section*{Machine de Turing de $\lambda$x,y[x - y]}

\subsection*{Entrée(s) :}

x et y sur la bande de la forme $|^{x+1}B|^{y+1}$

\subsection*{Sortie(s) :}

$\left\{
	\begin{array}{ll}
		x-y ~ si ~ x >= y\\
		\uparrow sinon
	\end{array}
\right.$

\subsection*{Principe :}

On parcourt les deux mots d'entrée un à un en remplaçant à chaque fois un symbole $|$ par un *.
Si le premier mot n'a plus de $|$ alors que le second en a toujours, on boucle.

\subsection*{Spécifications :}

Tant qu'on a encore des $|$ sur le mot de droite:
	\begin{itemize}
		\item On se déplace sur le dernier $|$ du mot de droite
		\item On le remplace par un *
		\item On va à gauche
		\item Si on lit un B alors on a plus de $|$ sur le mot de droite, on passe dans un état spécial
		\item On se remet sur le premier $|$ du mot de gauche
		\item Si il n'y a pas de $|$ sur le mot de gauche, on boucle
		\item Sinon on remplace le $|$ par un *
		\item Si on est dans l'état spécial, on s'arrête
	\end{itemize}

\subsection*{Machine de Turing :}

On se déplace jusqu'au dernier $|$ du mot de droite

\begin{tabular}{llll}
$q_0$ & $|$ & R & $q_0$ \\
$q_0$ & B & R & $q_1$ \\
$q_1$ & $|$ & R & $q_1$ \\
\end{tabular}\\

On se remet à gauche qu'on lise un B ou un *

\begin{tabular}{llll}
$q_1$ & B & L & $q_2$ \\
$q_1$ & * & L & $q_2$
\end{tabular}\\

On remplace $|$ par un *

\begin{tabular}{llll}
$q_2$ & $|$ & * & $q_2$
\end{tabular}\\

On va à gauche

\begin{tabular}{llll}
$q_2$ & * & L & $q_3$
\end{tabular}\\

Si on lit un B alors on a plus de $|$ sur le mot de droite donc on passe dans un état spécial $q_{y0}$ sinon on se déplace à gauche jusqu'à un B

\begin{tabular}{llll}
$q_3$ & B & L & $q_{y0}$ \\
$q_3$ & $|$ & L & $q_4$ \\
$q_4$ & $|$ & L & $q_4$ \\
$q_4$ & B & L & $q_5$
\end{tabular}\\

Si on est dans l'état spécial et qu'on lit un * alors on a pas assez de $|$ à gauche donc on boucle, sinon si on lit un $|$ on le remplace par un * et on termine

\begin{tabular}{llll}
$q_{y0}$ & * & * & $q_{y0}$ \\
$q_{y0}$ & $|$ & * & $q_{fin}$
\end{tabular}\\

Sinon on se remet tout à gauche

\begin{tabular}{llll}
$q_5$ & $|$ & L & $q_5$ \\
$q_5$ & B & R & $q_6$ \\
$q_5$ & * & R & $q_6$
\end{tabular}\\

On remplace le $|$ par * et on recommence

\begin{tabular}{llll}
$q_6$ & $|$ & * & $q_6$ \\
$q_6$ & * & R & $q_0$ \\
$q_6$ & B & B & $q_6$
\end{tabular}

\end{document}