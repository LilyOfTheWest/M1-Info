\documentclass{article}

\usepackage[utf8]{inputenc}
\usepackage[T1]{fontenc}
\usepackage[francais]{babel}
\usepackage{url}
\usepackage{color}
\usepackage{verbatim}
\usepackage{amsmath,amssymb,amsfonts}
\usepackage{graphicx}
\usepackage[french]{algorithm2e}
\usepackage{geometry}
\usepackage{enumitem}
\usepackage{listings}
\frenchbsetup{StandardLists=true}
\geometry{hmargin=2.5cm, vmargin=2.5cm}

\title{Introduction aux Réseaux TP3 : \\ Interconnexions de réseaux : le problème du routage}
\author{Line \bsc{POUVARET}, Mickaël \bsc{TURNEL}}
\date{2015-2016}


\begin{document}

\maketitle

\section*{2.1 Manipulation de la table de routage}

\begin{itemize}\renewcommand{\labelitemi}{$\bullet$}
\item a
\item a
\end{itemize}


\section*{2.2 Observation du routage automatique}

\begin{itemize}\renewcommand{\labelitemi}{$\bullet$}
\item b
\item a
\end{itemize}


\section*{2.3 Troisième manipulation}

\begin{itemize}\renewcommand{\labelitemi}{$\bullet$}
\item c
\item a
\end{itemize}


\subsection*{2.3.1 Routage statique}

\begin{itemize}\renewcommand{\labelitemi}{$\bullet$}
\item d
\item a
\end{itemize}


\subsection*{2.3.2 Routage automatique}

\begin{itemize}\renewcommand{\labelitemi}{$\bullet$}
\item e
\item a
\end{itemize}


\section*{2.4 Quatrième manipulation}

\begin{itemize}\renewcommand{\labelitemi}{$\bullet$}
\item f
\item e
\item a
\end{itemize}


\section*{2.5 Cinquième manipulation}

\begin{itemize}\renewcommand{\labelitemi}{$\bullet$}
\item g
\item e
\item a
\end{itemize}


\section*{3 Pour ceux qui veulent aller un peu plus loin}

\begin{itemize}\renewcommand{\labelitemi}{$\bullet$}
\item h
\item e
\item a
\end{itemize}



\end{document}