\documentclass{article}

\usepackage[utf8]{inputenc}
\usepackage[T1]{fontenc}
\usepackage[francais]{babel}
\usepackage{url}
\usepackage{color}
\usepackage{verbatim}
\usepackage{amsmath,amssymb,amsfonts}
\usepackage{graphicx}
\usepackage[french]{algorithm2e}
\usepackage{geometry}
\usepackage{enumitem}
\usepackage{listings}
\usepackage{listingsutf8}
\frenchbsetup{StandardLists=true}
\lstset{language=Java}
\lstset{
	breaklines=true, 
	showspaces=false, 
	keepspaces=true, 
	numbers=left, 
	frame=shadowbox, 
	keywordstyle=\color{blue},
	basicstyle=\ttfamily\small,
	commentstyle=\color{green}
}
\geometry{hmargin=2.5cm, vmargin=2.5cm}

\title{Base de Données : Compte-rendu du TP2\\Application "Zoo"}
\author{Line \bsc{POUVARET}, Mickaël \bsc{TURNEL}}
\date{2015-2016}

\begin{document}
\maketitle

\begin{itemize}\renewcommand{\labelitemi}{$\bullet$}
	\item Le calcul du nombre de maladies pour chaque animal doit être automatisé en fonction de l’ajout
ou de la suppression des maladies.

\lstset{language=SQL}
\begin{lstlisting}
CREATE or REPLACE TRIGGER MaladiesTrig
AFTER INSERT or DELETE on LesMaladies
FOR EACH ROW
BEGIN
	IF INSERTING THEN UPDATE LesAnimaux SET nb_maladies = nb_maladies+1 WHERE nomA=:new.nomA;
	ELSIF DELETING THEN UPDATE LesAnimaux SET nb_maladies = nb_maladies-1 WHERE nomA=:old.nomA;
	END IF;
END;
/
\end{lstlisting}

	\item Des animaux ne peuvent pas être placés dans une cage dont la fonction est incompatible avec
ces animaux. On prendra en compte le fait que des animaux peuvent être ajoutés, mais aussi
déplacés d’une cage.\\

\begin{lstlisting}
CREATE OR REPLACE TRIGGER DeplacerCageTrig
BEFORE INSERT OR UPDATE OF noCage on LesAnimaux
FOR EACH ROW
DECLARE
	fct VARCHAR2(20);
BEGIN
	SELECT fonction into fct FROM LesCages WHERE noCage = :new.noCage;
	IF (fct != :new.fonction_cage) THEN
		raise_application_error(-20001, 'cage incompatible');
	END IF;
EXCEPTION 
	WHEN NO_DATA_FOUND THEN
		raise_application_error(-20002, 'cage inexistante');
END;
/
\end{lstlisting}
	\item Des animaux ne peuvent pas être placés dans une cage non gardée. On prendra en compte le
fait que des animaux peuvent être ajoutés, mais aussi déplacés d’une cage.

\begin{lstlisting}
CREATE OR REPLACE TRIGGER CageGardeeTrig
BEFORE INSERT OR UPDATE OF noCage on LesAnimaux
FOR EACH ROW
DECLARE
	nb INTEGER;
BEGIN
	SELECT count(*) into nb FROM LesGardiens WHERE noCage = :new.noCage;
	IF (nb = 0) THEN
		raise_application_error(-20003, 'Cage non gardee');
	END IF;
EXCEPTION 
	WHEN NO_DATA_FOUND THEN
		raise_application_error(-20002, 'cage inexistante');
END;
/
\end{lstlisting}
	\item Des animaux de type différent ne peuvent pas cohabiter dans une même cage. On prendra en
compte le fait que des animaux peuvent être ajoutés, mais aussi déplacés d’une cage.
\begin{lstlisting}
CREATE OR REPLACE TRIGGER CohabiterCageTrig
AFTER INSERT OR UPDATE OF noCage on LesAnimaux
DECLARE
	nb INTEGER;
BEGIN
	SELECT count(*) into nb FROM(
		SELECT noCage, count(DISTINCT type_an)
		FROM LesAnimaux
		GROUP BY noCage
		HAVING COUNT (distinct type_an) > 1
	);
	IF (nb > 0) THEN
		raise_application_error(-20003, 'Des animaux de type differents ne peuvent cohabiter');
	END IF;
EXCEPTION 
	WHEN NO_DATA_FOUND THEN
		raise_application_error(-20002, 'Cage inexistante');
END;
/
\end{lstlisting}
\end{itemize}
\end{document}