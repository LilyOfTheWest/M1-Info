\documentclass{article}

\usepackage[utf8]{inputenc}
\usepackage[T1]{fontenc}
\usepackage[francais]{babel}
\usepackage{url}
\usepackage{color}
\usepackage{lmodern}
\usepackage{verbatim}
\usepackage{amsmath,amssymb,amsfonts}
\usepackage{mathrsfs}
\usepackage{graphicx}
\usepackage[french]{algorithm2e}
\usepackage{geometry}
\usepackage{enumitem}
\usepackage{listings}
\usepackage{listingsutf8}
\frenchbsetup{StandardLists=true}
\lstset{language=Java}
\lstset{
	breaklines=true, 
	showspaces=false, 
	keepspaces=true, 
	numbers=left, 
	frame=shadowbox, 
	keywordstyle=\color{blue},
	basicstyle=\ttfamily\small,
	commentstyle=\color{green}
}
\geometry{hmargin=2.5cm, vmargin=2.5cm}

\title{Base de Données}
\author{Line \bsc{POUVARET}}
\date{2015-2016}

\begin{document}
\maketitle

\section*{Bilan sur "les performances"}

\begin{itemize}\renewcommand{\labelitemi}{$\bullet$}
	\item Card(R) = 
	\begin{itemize}
		\item nb de tuples de la relation
		\item Soit e le nb de tuuples de R par page
	\end{itemize}
	
	\[NP(R) = \lceil\frac{Card(R)}{e}\rceil\]
	\item Card(I) = 
	\begin{itemize}
		\item nb valeur $\neq$ de l'attribut indexé
		\item Soit d le nb de tuples de I par page (d $\gg$ e)

	\[NP(I) = \lceil\frac{Card(I)}{d}\rceil\]
	\end{itemize}
\end{itemize}

\subsection*{Exemple}
Card(R) = 21

e = 5

NP(R) = 5

\subsection*{Remarque}
Card(I) = Card(R)

si l'attribut indexé est clé de R.

\begin{tabular}{|p{1,5cm}|p{1,5cm}|p{1,5cm}|p{1,5cm}|p{1,5cm}|p{1,5cm}|}
\hline
                  & Balayage "select * from R" & Recherche sur égalité "X=..." & Plage de valeur X > '...' & Insertion & delete \\
\hline
Sequentiel & NP(R) & $\frac{NP(R)}{2}$ en moyenne & NP(R) & Cst & NP(R) \\
\hline
Hashé & NP(R) & 1 & NP(R) & Cst & NP(R) \\
\hline
Index ou B-arbre & NP(R) & $\log_d$ (Card(R)) & $\log_d$ (Card(R)) + nb de pages où il y a des tuples vérifiant la propriété & Insertion dans le $B^{+}$ $\log_d$ (Card(R)) &  Insertion dans le $B^{+}$ $\log_d$ (Card(R)) \\
\hline
\end{tabular}

$\rightarrow$ Organisation $\phi$ des données est un paramètre important

$\rightarrow$ Indexation : 
\begin{itemize}
	\item rapidité d'accès
	\item mise à jour coûteuse
\end{itemize}

$\rightarrow$ Grouper les mises à jour (en présence d'index) pour éviter de perturber les lectures.

$\rightarrow$ Compromis lectures/mises à jour.

\section*{Implantation des opérateurs d'accès aux données}

$\rightarrow$ Opérations algébriques

$\rightarrow$ Opérateurs de tri (Order by/Group by)

$\rightarrow$ Agrégation (max, sum, count, min)

\begin{itemize}\renewcommand{\labelitemi}{$\bullet$}
	\item \underline{Selection} : $\sigma_{F}$(R)

clause "where" du SQL

Soit par accès séquentiel

Soit par un index

	\item \underline{Projection} : $\Pi_{y}$(R)

clause "select" de SQL



$\rightarrow$ Elimination des "doublons"
	\begin{itemize}
		\item $\theta$($n^{2}$) méthode naïve
		\item Utilisation d'une fonction de hashage $\rightarrow$ $\theta$(n)
	\end{itemize}

	\item \underline{Jointure naturelle} : R $\bowtie$ S = S $\bowtie$ R
		
	\begin{itemize}
		\item Pour une série de n jointures, on a (n-1)! séquence de jointures possibles
		\item Comment choisir la bonne séquence ?
	\end{itemize} 
\end{itemize}

$\rightarrow$ Influence de stockage

$\rightarrow$ Influence des algos

\subsection*{Algo boucles imbriquées :}  
R $\bowtie$ S             

(R.a = S.a)

\begin{algorithm}

T $\leftarrow$ $\varnothing$
\Pour{r $\in$ R}{
	\Pour{s $\in$ S}{
		\Si {r.a = s.a }{
			ajoute(r,s) à T;
		}
	}
}
\end{algorithm}

\subsection*{Remarque}
\begin{itemize} 
	\item Le résultat est dans l'ordre de la relation externe
	\item Coût nb de comp $\theta$($n^{2}$)

$\rightarrow$ Coût en E/S

Considérons une mémoire pouvant contenir b+2 pages de R

Coût en nb d'E/S

\[NP(R) + \lceil\frac{NP(R)}{b}\rceil * NP(S)\]

\subsection*{Remarque}

La jointure par boucles imbriquées n'est pas "symétrique".

$\rightarrow$ l'ordre du résultat
$\rightarrow$ Coût E/S

\subsection*{\underline{Algo Tri-Fusion}}

Procéder en 2 étapes :

\begin{enumerate}
	\item Trier R et S sur les attributs de jointure
	\item Fusion des tables triées (évaluation de la condition de jointure)
\end{enumerate}
\end{itemize}
	
\subsection*{Tri externe de R :}
NP(R) * $\log_b$(NP(R))

où b est le nb de pages de R en mémoire.

Pour la jointure Tri fusion :
\begin{itemize}
	\item Tri de R: NP(R) * $\log_b$(NP(R))
	\item Tri de S : NP(S) * $\log_b$(NP(S))
	\item Fusion : NP(R) + NP(S)
\end{itemize}

\subsection*{\underline{Jointure par Hash Code}}

($\theta$ condition soit une "égalité")

R $\bowtie$ S    

(R.a = S.a)

\begin{algorithm}
\Pour{chaque tuple r $\in$ R}{
	ajouter r à Table(h(r));
}

\Pour{chaque tuple s $\in$ S}{
	$E_{r}$ $\leftarrow$ Table(h(s));

	\Pour{chaque r $\in$ $E_{r}$}{
		Ajouter (r, s) à T;
	}
}
\end{algorithm}

Si les 2 tables sont déjà hashé :

NP(R) + NP(S)

S oui mais pas R

NP(R) + Card(S)

\section*{Estimation des coûts des $\neq$ chemins d'accès}

\begin{itemize}
	\item balayage séquentiel
	\item dichotomique (si triée)
	\item index
\end{itemize}

Pour une même requête, pour les différentes options d'évaluation (différents plans d'exécution), on estime un coût et on choisit l'option qui a le coût le plus faible.

En général, le coût est fonction de :
\begin{itemize}
	\item Coût en E/S 

$\rightarrow$ 1
	\item Coût CPU

 $\rightarrow$ $\frac{1}{100}$
	\item Coût de "com" si la BD est répartie 

$\rightarrow$ Coût CPU < Coût Com < Coût E/S
\end{itemize}

$\rightarrow$ Evaluer les cardinalités (nb de tuples) des relations initiales et intermédiaires

$\rightarrow$ On utilise des estimations basées sur des informations (statistique) que le SGBD maintient dans son catalogue.
\end{document}