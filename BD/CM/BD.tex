\documentclass{article}

\usepackage[utf8]{inputenc}
\usepackage[T1]{fontenc}
\usepackage[francais]{babel}
\usepackage{url}
\usepackage{color}
\usepackage{lmodern}
\usepackage{verbatim}
\usepackage{amsmath,amssymb,amsfonts}
\usepackage{mathrsfs}
\usepackage{multirow}
\usepackage{graphicx}
\usepackage[french]{algorithm2e}
\usepackage{geometry}
\usepackage{enumitem}
\usepackage{listings}
\usepackage{listingsutf8}
\frenchbsetup{StandardLists=true}
\lstset{language=Java}
\lstset{
	breaklines=true, 
	showspaces=false, 
	keepspaces=true, 
	numbers=left, 
	frame=shadowbox, 
	keywordstyle=\color{blue},
	basicstyle=\ttfamily\small,
	commentstyle=\color{green}
}
\geometry{hmargin=2.5cm, vmargin=2.5cm}

\title{Base de Données}
\author{Line \bsc{POUVARET}}
\date{2015-2016}

\begin{document}
\maketitle

\section*{Bilan sur "les performances"}

\begin{itemize}\renewcommand{\labelitemi}{$\bullet$}
	\item Card(R) = 
	\begin{itemize}
		\item nb de tuples de la relation
		\item Soit e le nb de tuuples de R par page
	\end{itemize}
	
	\[NP(R) = \lceil\frac{Card(R)}{e}\rceil\]
	\item Card(I) = 
	\begin{itemize}
		\item nb valeur $\neq$ de l'attribut indexé
		\item Soit d le nb de tuples de I par page (d $\gg$ e)

	\[NP(I) = \lceil\frac{Card(I)}{d}\rceil\]
	\end{itemize}
\end{itemize}

\subsection*{Exemple}
Card(R) = 21

e = 5

NP(R) = 5

\subsection*{Remarque}
Card(I) = Card(R)

si l'attribut indexé est clé de R.

\begin{tabular}{|p{1,5cm}|p{1,5cm}|p{1,5cm}|p{1,5cm}|p{1,5cm}|p{1,5cm}|}
\hline
                  & Balayage "select * from R" & Recherche sur égalité "X=..." & Plage de valeur X > '...' & Insertion & delete \\
\hline
Sequentiel & NP(R) & $\frac{NP(R)}{2}$ en moyenne & NP(R) & Cst & NP(R) \\
\hline
Hashé & NP(R) & 1 & NP(R) & Cst & NP(R) \\
\hline
Index ou B-arbre & NP(R) & $\log_d$ (Card(R)) & $\log_d$ (Card(R)) + nb de pages où il y a des tuples vérifiant la propriété & Insertion dans le $B^{+}$ $\log_d$ (Card(R)) &  Insertion dans le $B^{+}$ $\log_d$ (Card(R)) \\
\hline
\end{tabular}

$\rightarrow$ Organisation $\phi$ des données est un paramètre important

$\rightarrow$ Indexation : 
\begin{itemize}
	\item rapidité d'accès
	\item mise à jour coûteuse
\end{itemize}

$\rightarrow$ Grouper les mises à jour (en présence d'index) pour éviter de perturber les lectures.

$\rightarrow$ Compromis lectures/mises à jour.

\section*{Implantation des opérateurs d'accès aux données}

$\rightarrow$ Opérations algébriques

$\rightarrow$ Opérateurs de tri (Order by/Group by)

$\rightarrow$ Agrégation (max, sum, count, min)

\begin{itemize}\renewcommand{\labelitemi}{$\bullet$}
	\item \underline{Selection} : $\sigma_{F}$(R)

clause "where" du SQL

Soit par accès séquentiel

Soit par un index

	\item \underline{Projection} : $\Pi_{y}$(R)

clause "select" de SQL



$\rightarrow$ Elimination des "doublons"
	\begin{itemize}
		\item $\theta$($n^{2}$) méthode naïve
		\item Utilisation d'une fonction de hashage $\rightarrow$ $\theta$(n)
	\end{itemize}

	\item \underline{Jointure naturelle} : R $\bowtie$ S = S $\bowtie$ R
		
	\begin{itemize}
		\item Pour une série de n jointures, on a (n-1)! séquence de jointures possibles
		\item Comment choisir la bonne séquence ?
	\end{itemize} 
\end{itemize}

$\rightarrow$ Influence de stockage

$\rightarrow$ Influence des algos

\subsection*{Algo boucles imbriquées :}  
R $\bowtie$ S             

(R.a = S.a)

\begin{algorithm}

T $\leftarrow$ $\varnothing$
\Pour{r $\in$ R}{
	\Pour{s $\in$ S}{
		\Si {r.a = s.a }{
			ajoute(r,s) à T;
		}
	}
}
\end{algorithm}

\subsection*{Remarque}
\begin{itemize} 
	\item Le résultat est dans l'ordre de la relation externe
	\item Coût nb de comp $\theta$($n^{2}$)

$\rightarrow$ Coût en E/S

Considérons une mémoire pouvant contenir b+2 pages de R

Coût en nb d'E/S

\[NP(R) + \lceil\frac{NP(R)}{b}\rceil * NP(S)\]

\subsection*{Remarque}

La jointure par boucles imbriquées n'est pas "symétrique".

$\rightarrow$ l'ordre du résultat
$\rightarrow$ Coût E/S

\subsection*{\underline{Algo Tri-Fusion}}

Procéder en 2 étapes :

\begin{enumerate}
	\item Trier R et S sur les attributs de jointure
	\item Fusion des tables triées (évaluation de la condition de jointure)
\end{enumerate}
\end{itemize}
	
\subsection*{Tri externe de R :}
NP(R) * $\log_b$(NP(R))

où b est le nb de pages de R en mémoire.

Pour la jointure Tri fusion :
\begin{itemize}
	\item Tri de R: NP(R) * $\log_b$(NP(R))
	\item Tri de S : NP(S) * $\log_b$(NP(S))
	\item Fusion : NP(R) + NP(S)
\end{itemize}

\subsection*{\underline{Jointure par Hash Code}}

($\theta$ condition soit une "égalité")

R $\bowtie$ S    

(R.a = S.a)

\begin{algorithm}
\Pour{chaque tuple r $\in$ R}{
	ajouter r à Table(h(r));
}

\Pour{chaque tuple s $\in$ S}{
	$E_{r}$ $\leftarrow$ Table(h(s));

	\Pour{chaque r $\in$ $E_{r}$}{
		Ajouter (r, s) à T;
	}
}
\end{algorithm}

Si les 2 tables sont déjà hashé :

NP(R) + NP(S)

S oui mais pas R

NP(R) + Card(S)

\section*{Estimation des coûts des $\neq$ chemins d'accès}

\begin{itemize}
	\item balayage séquentiel
	\item dichotomique (si triée)
	\item index
\end{itemize}

Pour une même requête, pour les différentes options d'évaluation (différents plans d'exécution), on estime un coût et on choisit l'option qui a le coût le plus faible.

En général, le coût est fonction de :
\begin{itemize}
	\item Coût en E/S 

$\rightarrow$ 1
	\item Coût CPU

 $\rightarrow$ $\frac{1}{100}$
	\item Coût de "com" si la BD est répartie 

$\rightarrow$ Coût CPU < Coût Com < Coût E/S
\end{itemize}

$\rightarrow$ Evaluer les cardinalités (nb de tuples) des relations initiales et intermédiaires

$\rightarrow$ On utilise des estimations basées sur des informations (statistique) que le SGBD maintient dans son catalogue.\\

CA = $\underbrace{\sigma*NbP}_{E/S}+\underbrace{\mu*NbT}_{calcul}+\underbrace{\nu*NbM}_{comm}$\\

Info "stat" que le SGBD maintient dans un "catalogue".\\

\subsection*{\underline{Informations statistiques}}

$\rightarrow$ Pour chaque relation R :
\begin{itemize}\renewcommand{\labelitemi}{$\bullet$}
	\item Card(R) : la cardinalité de R est le nombre de tuples de la relation.
	\item NP(R) : le nombre de pages nécessaires pour contenir R
	\item FacP(R) = le facteur de mise en "blocs" de R. Le nb de tuples de R dans une page.
\end{itemize}

NP(R) = $\lceil$ $\frac{Card(R)}{FacP(R)}$$\rceil$

$\rightarrow$ Pour chaque attribut A de la relation R :

\begin{itemize}\renewcommand{\labelitemi}{$\bullet$}
	\item $Card_{R}$(A) = la cardinalité de A dans R. Le nombre de valeurs distinctes qui apparaissent pour A dans la relation R.
	\item $Min_{R}$(A), $Max_{R}$(A) $\rightarrow$ les valeurs Min et Max pour l'attribut A dans la relation R.
\end{itemize}

$\rightarrow$ Pour chaque index I 
\begin{itemize}\renewcommand{\labelitemi}{$\bullet$}
	\item NP(I) : le nombre de pages de l'index
	\item Prof(I) : le nombre de niveaux de I
\end{itemize}

\subsection*{\underline{Estimation des cardinalités}}

\begin{itemize}\renewcommand{\labelitemi}{$\bullet$}
	\item Sélection $\sigma_{F}$(R)

On note SF(F) le facteur de sélectivité de F : 

Card($\sigma_{F}$(R)) = SF(F) * Card(R)

SF(F) $\leq$ 1 et Card($\sigma_{F}$(R)) $\leq$  Card(R)

Pour un attribut A, on suppose que les valeurs de A sont équi-probables dans R et qu'il y a au moins une valeur qui satisfait la condition.


\begin{tabular}{|p{2cm}|p{4cm}|p{4cm}|}
\hline
F & SF(F) & Card($\sigma_{F}$(R)) \\
\hline
\multirow{2}{3cm}{A=x} & $\frac{1}{Card_{R}(A)}$ & 1 \\
\cline{2-3}
& $\frac{1}{Card_{R}(A)}$ & $\frac{1}{Card_{R}(A)} * Card(R)$ \\
\hline
A > c & $\frac{Max_{R}(A)-C}{Max_{R}(A)-Min_{R}(A)}$ & SF(F) * Card(R) \\
\hline
A < c & $\frac{c - Min_{R}(A)}{Max_{R}(A) - Min{R}(A)}$ & -- \\
\hline
$c_{1}$ < A $c_{2}$ & $\frac{c_{1} - c_{2}}{Max_{R}(A) - Min_{R}(A)}$ & -- \\
\hline 
$F_{1} \wedge F_{2}$ & $SF(F_{1}) * SF(F_{2})$ & -- \\
\hline
$F_{1} \vee F_{2}$ & $SF(F_{1}) + SF(F_{2}) - SF(F_{1}) * SF(F_{2})$ & -- \\
\hline
$\neg F$ & 1-SF(F) & -- \\
\hline
$A_{1} = A_{2}$ & $\frac{1}{Max(Card(A_{1}), Card(A_{2}))}$ & -- \\
\hline
\end{tabular}

	\item \underline{Produit cartésien}

R x S

Card(R x S) = Card(R) * Card(S)

	\item \underline{Jointure}

T = R $\bowtie$ S               $A_{1} = A_{2}$

où $A_{1} \in Att(R)$

$A_{2} \in Att(S)$

On a Card(T) $\leq$ Card(R) * Card(S)

	\begin{itemize}
		\item Si la condition de jointure porte sur une clé de l'une des relations (par exemple R) alors on a :

Card(T) $\leq$ Card(S)

(1 tuple de S peut joindre au max 1 tuple de R)

		\item Dans le cas général, la cardinalité de la jointure est la cardinalité du produit cartésien suivi d'une sélection

$Card(T) = \frac{1}{Max(Card(A_{1}), Card(A_{2}))} * Card(R)$

	\end{itemize}

	\item \underline{Union ensembliste}

T = R $\cup$ S 

Card(T) $\neq$ Card(R) + Card(S)

Max(Card(R), Card(S)) $\leq$ Card(T) $\leq$ Card(R) + Card(S)

	\item \underline{Intersection}

T = R $\cap$ S

0 $\leq$ Card(T) $\leq$ Min(Card(R), Card(S))

	\item \underline{Différence}

T = R - S

0 $\leq$ Card(T) $\leq$ Card(R)

\end{itemize}

\subsection*{\underline{Estimation des coûts}}

On se limite au coût d'accès (nb de pages)

\begin{itemize}\renewcommand{\labelitemi}{$\bullet$}
	\item  Sélection $\sigma_{F}(R)$

$\rightarrow$ Recherche linéaire : NP(R)  (Full scan)\\

\underline{Remarque :} Si F est du type A = x où A est une clé alors : $\frac{NR(R)}{2}$ coût moyen\\

$\rightarrow$ Si F est de type (A = x) où A est une clé de hachage CA = 1 (Coût d'accès)

$\rightarrow$ Egalité avec Index multiniveau

(A = x)

Prof(I) + 1 (Si A est une clé de R)

Prof(I) + SF(F) * Card(R)\\

$\rightarrow$ Inégalité avec Index :

Prof(I) + $\frac{NP(R)}{2}$ (en moyenne, si F porte sur la clé)

Prof(I) + SF(F) * Card(R)\\

\underline{Exemple :}\\

Card(R) = 3000

FacP(R) = 30

$\rightarrow$ NP(R) = 100\\

I est un index sur A $\in$ Att(R)

Prof(I) = 2

$Card_{R}(A) = 500$

$Min_{R}(A) = 10 000$

$Max_{R}(A) = 50 000$\\

R' = $\sigma_{A=20000}(R)$

Card(R') = $\underbrace{SF(A=20000)}_{\frac{1}{500}} *\underbrace{Card(R)}_{3000}$ = 6

\begin{itemize}
	\item Recherche linéaire : Ca = NP(R) = 100
	\item Index : CA = Prof(I) + SF(F) * Card(R) = 2 + 6 = 8
\end{itemize}

R" : $\sigma_{A > 20000}(R)$

Card(R") = SF(F) * Card(R)

Card(R") = $\frac{Max_{R}(A) - 20000}{Max_{R}(A)-Min_{R}(A)} * Card(R)$

Card(R") = $\frac{50000-20000}{50000-10000}*3000$

Card(R") = 2250

\begin{itemize}
	\item Recherche linéaire : CA = NP(R) = 100
	\item Index : CA = Prof(I) + SF(F) * Card(R) = 2 + 2250 

CA = 2252
\end{itemize}

\end{itemize}

\section*{\underline{SGBD} (Oracle)}

2 modes d'optimisation : 
\begin{itemize}\renewcommand{\labelitemi}{$\bullet$}
	\item orienté règles (RULE)
	\item orienté coût : Paramètre OPTIMIZER\_MODE (CHOOSE)

DBMS\_STATS (PL/SQL)

L'administrateur choisit à quelle fréquence les statistiques doivent être calculées.
\end{itemize}

$\rightarrow$ EXPLAIN\_PLAN : permet de connaître le plan d'exécution d'une requête 

$\rightarrow$ "hints" : 

\lstset{language=SQL, frame=none, numbers=none}
\begin{lstlisting}
SELECT /* + Index(genreIndex) */ nom, prenom
FROM Etudiant
WHERE genre='M';
\end{lstlisting}

\end{document}