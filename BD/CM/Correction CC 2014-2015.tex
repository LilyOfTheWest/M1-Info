\documentclass{article}

\usepackage[utf8]{inputenc}
\usepackage[T1]{fontenc}
\usepackage[francais]{babel}
\usepackage{url}
\usepackage{color}
\usepackage{lmodern}
\usepackage{verbatim}
\usepackage{amsmath,amssymb,amsfonts}
\usepackage{mathrsfs}
\usepackage{multirow}
\usepackage{graphicx}
\usepackage[french]{algorithm2e}
\usepackage{geometry}
\usepackage{enumitem}
\usepackage{listings}
\usepackage{listingsutf8}
\usepackage{qtree}
\frenchbsetup{StandardLists=true}
\lstset{language=Java}
\lstset{
	breaklines=true, 
	showspaces=false, 
	keepspaces=true, 
	numbers=left, 
	frame=shadowbox, 
	keywordstyle=\color{blue},
	basicstyle=\ttfamily\small,
	commentstyle=\color{green}
}
\geometry{hmargin=2.5cm, vmargin=2.5cm}

\title{Base de Données\\Correction CC 2014-2015}
\author{Line \bsc{POUVARET}}
\date{2015-2016}

\begin{document}
\maketitle


\section*{III – Transaction}
\subsection*{Q3)} 
\begin{itemize}\renewcommand{\labelitemi}{$\bullet$}
	\item 1 $\rightarrow$ 0
	\item 2 $\rightarrow$ 2
	\item 3 $\rightarrow$ 0
	\item 4 $\rightarrow$ 5
	\item 5 $\rightarrow$ mis en attente (verrou par S2) puis 0 row modified (puisque il ne trouve pas de d5)
	\item 6 $\rightarrow$ 0
\end{itemize}

\subsection*{Q4)}
	Sérialisable car on a exécuté tout S2 puis tout S1.

\section*{IV- Optimisation de requêtes}
\subsection*{Q5)}
10 valeurs possibles pour un attribut (donné par défaut).

$ Card(\sigma_{prix<2000}(LesStations) = Card(LesStations) x SF(\sigma_{prix<2000}) $\\

$ SF(\sigma_{prix<2000}) = \frac{200-ValMin_{LesStations}(prix)}{ValMax_{LesStations}(prix)-ValMin_{LesStations}(Prix)} $\\

Valeur arbitraire : $ \frac{1}{10} $\\

$ Card(\sigma_{p<2000}(LS)) = 50 $\\

$ Card(\sigma_{nbPlaces=2}(LesSejours)) = 100 000 * \frac{1}{30} = Card(LesSejours * SF(\sigma_{nbPlaces=2})) $\\

\Tree [.{$\bowtie$} [.{$\bowtie$} [.{$\sigma_{prix<2000}$} LesStations ] [.{$\sigma_{nbPlaces=2}$} LesSejours ] ] [.{$\sigma_{Solde >= 10000}$} LesClients ] ]

\end{document}