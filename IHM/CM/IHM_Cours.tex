\documentclass{article}

\usepackage[utf8]{inputenc}
\usepackage[T1]{fontenc}
\usepackage[francais]{babel}
\usepackage{url}
\usepackage{color}
\usepackage{verbatim}
\usepackage{amsmath,amssymb,amsfonts}
\usepackage{graphicx}
\usepackage[french]{algorithm2e}
\usepackage{geometry}
\usepackage{caption}
\captionsetup[figure]{slc=on}
\usepackage{enumitem}
\usepackage{listings}
\usepackage{listingsutf8}
\frenchbsetup{StandardLists=true}
\lstset{
	breaklines=true, 
	showspaces=false, 
	keepspaces=true, 
	numbers=left,
	frame=shadowbox, 
	keywordstyle=\color{blue},
	basicstyle=\ttfamily\small,
	commentstyle=\color{green}
}
\geometry{hmargin=2.5cm, vmargin=2.5cm}

\title{\textbf{Ergonomie pour les Interfaces Homme-Machine}}
\author{Line \bsc{POUVARET}}
\date{2015-2016}

\begin{document}
\maketitle

\section{Définition d'une IHM}
Les IHM définissent les moyens et outils mis en oeuvre afin qu'un humain puisse contrôler et communiquer avec une machine.\\

\subsection{Les ordinateurs}
Ordinateur : unité de calcul + unité de stockage

Machine automatique de traitement de 'information obéissant à des programmes formés par une suite d'opérations arithmétiques et logiques.


\end{document}