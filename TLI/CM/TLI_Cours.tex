\documentclass{report}

\usepackage[utf8]{inputenc}
\usepackage[T1]{fontenc}
\usepackage[francais]{babel}
\usepackage{url}
\usepackage{color}
\usepackage{verbatim}
\usepackage{amsmath,amssymb,amsfonts}
\usepackage{graphicx}
\usepackage[french]{algorithm2e}
\usepackage{geometry}
\usepackage{caption}
\captionsetup[figure]{slc=on}
\usepackage{enumitem}
\usepackage{listings}
\usepackage{listingsutf8}
\frenchbsetup{StandardLists=true}
\lstset{
	breaklines=true, 
	showspaces=false, 
	keepspaces=true, 
	numbers=left,
	frame=shadowbox, 
	keywordstyle=\color{blue},
	basicstyle=\ttfamily\small,
	commentstyle=\color{green}
}
\geometry{hmargin=2.5cm, vmargin=2.5cm}

\title{\textbf{Techniques pour les Logiciels Interactifs}}
\author{Line \bsc{POUVARET}}
\date{2015-2016}

\begin{document}
\maketitle

\chapter{Introduction à l'Interaction Homme-Machine}

cours disponible sur http://iihm.imag.fr/blanch/M1

\section{L'interaction}

Interface d'un logiciel = interface utilisateur.

IHM = interaction homme-machine

\subsection{Définitions}


\begin{itemize}\renewcommand{\labelitemi}{$\bullet$}	
	\item Interface : Frontière entre le monde physique et le monde numérique
	\item Interaction : Échange continu (boucle). Agit par le biais d'interface sur le monde numérique.
	\item Interaction Homme-Machine : Évaluation et Conception vont de paire.
	
	La mise en œuvre (= réalisation) correspond au sujet du cours.
\end{itemize}

\subsection{Styles d'interaction}
\subsubsection{La ligne de commande}

read-eval-print-loop : l'interpréteur se met en attente du texte (sur l'entrée standard)\\

read : récupère l'entrée.

eval : interprète et analyse le résultat.

print : affiche le résultat.\\

Syntaxe + Sémantique doit être connue de l'utilisateur.

\subsubsection{Les menus/formulaires}
Arrivée des outils informatiques pour des non informaticiens.

Choix parmi des fonctions programmées.

\subsubsection{La manipulation directe ("direct manipulation"}
B. Schneiderman

Premiers ordinateurs personnels : Xerox Star et Xerox Lisa (avec fonctions de drag and drop)

\chapter{Principes pour le développement de logiciel interactif}
\section{Element d'architecture}
\begin{itemize}
	\item Séparation de l'interface\\
	
Principe 1 : sépaer Noyau Fonctionnel (quoi) et IU (comment)

Principe 2 : NF indépendant de IU

$\rightarrow$ Minimiser les dépendances

Principe 3 : NF conçu pour l'interaction 

	\begin{itemize}
		\item la notification
		\item en prévention des erreurs
		\item annulation
	\end{itemize}
	
	\item Modèle de Seeheim
\end{itemize}

\end{document}